% Generated by Sphinx.
\def\sphinxdocclass{report}
\documentclass[letterpaper,10pt,english]{sphinxmanual}
\usepackage[utf8]{inputenc}
\DeclareUnicodeCharacter{00A0}{\nobreakspace}
\usepackage{cmap}
\usepackage[T1]{fontenc}
\usepackage{amsfonts}
\usepackage{babel}
\usepackage{times}
\usepackage[Bjarne]{fncychap}
\usepackage{longtable}
\usepackage{sphinx}
\usepackage{multirow}
\usepackage{eqparbox}


\addto\captionsenglish{\renewcommand{\figurename}{Fig. }}
\addto\captionsenglish{\renewcommand{\tablename}{Table }}
\SetupFloatingEnvironment{literal-block}{name=Listing }



\title{an\_example\_pypi\_project Documentation}
\date{September 08, 2017}
\release{0.0.1}
\author{Abolfazl}
\newcommand{\sphinxlogo}{}
\renewcommand{\releasename}{Release}

\makeindex

\makeatletter
\def\PYG@reset{\let\PYG@it=\relax \let\PYG@bf=\relax%
    \let\PYG@ul=\relax \let\PYG@tc=\relax%
    \let\PYG@bc=\relax \let\PYG@ff=\relax}
\def\PYG@tok#1{\csname PYG@tok@#1\endcsname}
\def\PYG@toks#1+{\ifx\relax#1\empty\else%
    \PYG@tok{#1}\expandafter\PYG@toks\fi}
\def\PYG@do#1{\PYG@bc{\PYG@tc{\PYG@ul{%
    \PYG@it{\PYG@bf{\PYG@ff{#1}}}}}}}
\def\PYG#1#2{\PYG@reset\PYG@toks#1+\relax+\PYG@do{#2}}

\expandafter\def\csname PYG@tok@gd\endcsname{\def\PYG@tc##1{\textcolor[rgb]{0.63,0.00,0.00}{##1}}}
\expandafter\def\csname PYG@tok@gu\endcsname{\let\PYG@bf=\textbf\def\PYG@tc##1{\textcolor[rgb]{0.50,0.00,0.50}{##1}}}
\expandafter\def\csname PYG@tok@gt\endcsname{\def\PYG@tc##1{\textcolor[rgb]{0.00,0.27,0.87}{##1}}}
\expandafter\def\csname PYG@tok@gs\endcsname{\let\PYG@bf=\textbf}
\expandafter\def\csname PYG@tok@gr\endcsname{\def\PYG@tc##1{\textcolor[rgb]{1.00,0.00,0.00}{##1}}}
\expandafter\def\csname PYG@tok@cm\endcsname{\let\PYG@it=\textit\def\PYG@tc##1{\textcolor[rgb]{0.25,0.50,0.56}{##1}}}
\expandafter\def\csname PYG@tok@vg\endcsname{\def\PYG@tc##1{\textcolor[rgb]{0.73,0.38,0.84}{##1}}}
\expandafter\def\csname PYG@tok@vi\endcsname{\def\PYG@tc##1{\textcolor[rgb]{0.73,0.38,0.84}{##1}}}
\expandafter\def\csname PYG@tok@vm\endcsname{\def\PYG@tc##1{\textcolor[rgb]{0.73,0.38,0.84}{##1}}}
\expandafter\def\csname PYG@tok@mh\endcsname{\def\PYG@tc##1{\textcolor[rgb]{0.13,0.50,0.31}{##1}}}
\expandafter\def\csname PYG@tok@cs\endcsname{\def\PYG@tc##1{\textcolor[rgb]{0.25,0.50,0.56}{##1}}\def\PYG@bc##1{\setlength{\fboxsep}{0pt}\colorbox[rgb]{1.00,0.94,0.94}{\strut ##1}}}
\expandafter\def\csname PYG@tok@ge\endcsname{\let\PYG@it=\textit}
\expandafter\def\csname PYG@tok@vc\endcsname{\def\PYG@tc##1{\textcolor[rgb]{0.73,0.38,0.84}{##1}}}
\expandafter\def\csname PYG@tok@il\endcsname{\def\PYG@tc##1{\textcolor[rgb]{0.13,0.50,0.31}{##1}}}
\expandafter\def\csname PYG@tok@go\endcsname{\def\PYG@tc##1{\textcolor[rgb]{0.20,0.20,0.20}{##1}}}
\expandafter\def\csname PYG@tok@cp\endcsname{\def\PYG@tc##1{\textcolor[rgb]{0.00,0.44,0.13}{##1}}}
\expandafter\def\csname PYG@tok@gi\endcsname{\def\PYG@tc##1{\textcolor[rgb]{0.00,0.63,0.00}{##1}}}
\expandafter\def\csname PYG@tok@gh\endcsname{\let\PYG@bf=\textbf\def\PYG@tc##1{\textcolor[rgb]{0.00,0.00,0.50}{##1}}}
\expandafter\def\csname PYG@tok@ni\endcsname{\let\PYG@bf=\textbf\def\PYG@tc##1{\textcolor[rgb]{0.84,0.33,0.22}{##1}}}
\expandafter\def\csname PYG@tok@nl\endcsname{\let\PYG@bf=\textbf\def\PYG@tc##1{\textcolor[rgb]{0.00,0.13,0.44}{##1}}}
\expandafter\def\csname PYG@tok@nn\endcsname{\let\PYG@bf=\textbf\def\PYG@tc##1{\textcolor[rgb]{0.05,0.52,0.71}{##1}}}
\expandafter\def\csname PYG@tok@no\endcsname{\def\PYG@tc##1{\textcolor[rgb]{0.38,0.68,0.84}{##1}}}
\expandafter\def\csname PYG@tok@na\endcsname{\def\PYG@tc##1{\textcolor[rgb]{0.25,0.44,0.63}{##1}}}
\expandafter\def\csname PYG@tok@nb\endcsname{\def\PYG@tc##1{\textcolor[rgb]{0.00,0.44,0.13}{##1}}}
\expandafter\def\csname PYG@tok@nc\endcsname{\let\PYG@bf=\textbf\def\PYG@tc##1{\textcolor[rgb]{0.05,0.52,0.71}{##1}}}
\expandafter\def\csname PYG@tok@nd\endcsname{\let\PYG@bf=\textbf\def\PYG@tc##1{\textcolor[rgb]{0.33,0.33,0.33}{##1}}}
\expandafter\def\csname PYG@tok@ne\endcsname{\def\PYG@tc##1{\textcolor[rgb]{0.00,0.44,0.13}{##1}}}
\expandafter\def\csname PYG@tok@nf\endcsname{\def\PYG@tc##1{\textcolor[rgb]{0.02,0.16,0.49}{##1}}}
\expandafter\def\csname PYG@tok@si\endcsname{\let\PYG@it=\textit\def\PYG@tc##1{\textcolor[rgb]{0.44,0.63,0.82}{##1}}}
\expandafter\def\csname PYG@tok@s2\endcsname{\def\PYG@tc##1{\textcolor[rgb]{0.25,0.44,0.63}{##1}}}
\expandafter\def\csname PYG@tok@nt\endcsname{\let\PYG@bf=\textbf\def\PYG@tc##1{\textcolor[rgb]{0.02,0.16,0.45}{##1}}}
\expandafter\def\csname PYG@tok@nv\endcsname{\def\PYG@tc##1{\textcolor[rgb]{0.73,0.38,0.84}{##1}}}
\expandafter\def\csname PYG@tok@s1\endcsname{\def\PYG@tc##1{\textcolor[rgb]{0.25,0.44,0.63}{##1}}}
\expandafter\def\csname PYG@tok@dl\endcsname{\def\PYG@tc##1{\textcolor[rgb]{0.25,0.44,0.63}{##1}}}
\expandafter\def\csname PYG@tok@ch\endcsname{\let\PYG@it=\textit\def\PYG@tc##1{\textcolor[rgb]{0.25,0.50,0.56}{##1}}}
\expandafter\def\csname PYG@tok@m\endcsname{\def\PYG@tc##1{\textcolor[rgb]{0.13,0.50,0.31}{##1}}}
\expandafter\def\csname PYG@tok@gp\endcsname{\let\PYG@bf=\textbf\def\PYG@tc##1{\textcolor[rgb]{0.78,0.36,0.04}{##1}}}
\expandafter\def\csname PYG@tok@sh\endcsname{\def\PYG@tc##1{\textcolor[rgb]{0.25,0.44,0.63}{##1}}}
\expandafter\def\csname PYG@tok@ow\endcsname{\let\PYG@bf=\textbf\def\PYG@tc##1{\textcolor[rgb]{0.00,0.44,0.13}{##1}}}
\expandafter\def\csname PYG@tok@sx\endcsname{\def\PYG@tc##1{\textcolor[rgb]{0.78,0.36,0.04}{##1}}}
\expandafter\def\csname PYG@tok@bp\endcsname{\def\PYG@tc##1{\textcolor[rgb]{0.00,0.44,0.13}{##1}}}
\expandafter\def\csname PYG@tok@c1\endcsname{\let\PYG@it=\textit\def\PYG@tc##1{\textcolor[rgb]{0.25,0.50,0.56}{##1}}}
\expandafter\def\csname PYG@tok@fm\endcsname{\def\PYG@tc##1{\textcolor[rgb]{0.02,0.16,0.49}{##1}}}
\expandafter\def\csname PYG@tok@o\endcsname{\def\PYG@tc##1{\textcolor[rgb]{0.40,0.40,0.40}{##1}}}
\expandafter\def\csname PYG@tok@kc\endcsname{\let\PYG@bf=\textbf\def\PYG@tc##1{\textcolor[rgb]{0.00,0.44,0.13}{##1}}}
\expandafter\def\csname PYG@tok@c\endcsname{\let\PYG@it=\textit\def\PYG@tc##1{\textcolor[rgb]{0.25,0.50,0.56}{##1}}}
\expandafter\def\csname PYG@tok@mf\endcsname{\def\PYG@tc##1{\textcolor[rgb]{0.13,0.50,0.31}{##1}}}
\expandafter\def\csname PYG@tok@err\endcsname{\def\PYG@bc##1{\setlength{\fboxsep}{0pt}\fcolorbox[rgb]{1.00,0.00,0.00}{1,1,1}{\strut ##1}}}
\expandafter\def\csname PYG@tok@mb\endcsname{\def\PYG@tc##1{\textcolor[rgb]{0.13,0.50,0.31}{##1}}}
\expandafter\def\csname PYG@tok@ss\endcsname{\def\PYG@tc##1{\textcolor[rgb]{0.32,0.47,0.09}{##1}}}
\expandafter\def\csname PYG@tok@sr\endcsname{\def\PYG@tc##1{\textcolor[rgb]{0.14,0.33,0.53}{##1}}}
\expandafter\def\csname PYG@tok@mo\endcsname{\def\PYG@tc##1{\textcolor[rgb]{0.13,0.50,0.31}{##1}}}
\expandafter\def\csname PYG@tok@kd\endcsname{\let\PYG@bf=\textbf\def\PYG@tc##1{\textcolor[rgb]{0.00,0.44,0.13}{##1}}}
\expandafter\def\csname PYG@tok@mi\endcsname{\def\PYG@tc##1{\textcolor[rgb]{0.13,0.50,0.31}{##1}}}
\expandafter\def\csname PYG@tok@kn\endcsname{\let\PYG@bf=\textbf\def\PYG@tc##1{\textcolor[rgb]{0.00,0.44,0.13}{##1}}}
\expandafter\def\csname PYG@tok@cpf\endcsname{\let\PYG@it=\textit\def\PYG@tc##1{\textcolor[rgb]{0.25,0.50,0.56}{##1}}}
\expandafter\def\csname PYG@tok@kr\endcsname{\let\PYG@bf=\textbf\def\PYG@tc##1{\textcolor[rgb]{0.00,0.44,0.13}{##1}}}
\expandafter\def\csname PYG@tok@s\endcsname{\def\PYG@tc##1{\textcolor[rgb]{0.25,0.44,0.63}{##1}}}
\expandafter\def\csname PYG@tok@kp\endcsname{\def\PYG@tc##1{\textcolor[rgb]{0.00,0.44,0.13}{##1}}}
\expandafter\def\csname PYG@tok@w\endcsname{\def\PYG@tc##1{\textcolor[rgb]{0.73,0.73,0.73}{##1}}}
\expandafter\def\csname PYG@tok@kt\endcsname{\def\PYG@tc##1{\textcolor[rgb]{0.56,0.13,0.00}{##1}}}
\expandafter\def\csname PYG@tok@sc\endcsname{\def\PYG@tc##1{\textcolor[rgb]{0.25,0.44,0.63}{##1}}}
\expandafter\def\csname PYG@tok@sb\endcsname{\def\PYG@tc##1{\textcolor[rgb]{0.25,0.44,0.63}{##1}}}
\expandafter\def\csname PYG@tok@sa\endcsname{\def\PYG@tc##1{\textcolor[rgb]{0.25,0.44,0.63}{##1}}}
\expandafter\def\csname PYG@tok@k\endcsname{\let\PYG@bf=\textbf\def\PYG@tc##1{\textcolor[rgb]{0.00,0.44,0.13}{##1}}}
\expandafter\def\csname PYG@tok@se\endcsname{\let\PYG@bf=\textbf\def\PYG@tc##1{\textcolor[rgb]{0.25,0.44,0.63}{##1}}}
\expandafter\def\csname PYG@tok@sd\endcsname{\let\PYG@it=\textit\def\PYG@tc##1{\textcolor[rgb]{0.25,0.44,0.63}{##1}}}

\def\PYGZbs{\char`\\}
\def\PYGZus{\char`\_}
\def\PYGZob{\char`\{}
\def\PYGZcb{\char`\}}
\def\PYGZca{\char`\^}
\def\PYGZam{\char`\&}
\def\PYGZlt{\char`\<}
\def\PYGZgt{\char`\>}
\def\PYGZsh{\char`\#}
\def\PYGZpc{\char`\%}
\def\PYGZdl{\char`\$}
\def\PYGZhy{\char`\-}
\def\PYGZsq{\char`\'}
\def\PYGZdq{\char`\"}
\def\PYGZti{\char`\~}
% for compatibility with earlier versions
\def\PYGZat{@}
\def\PYGZlb{[}
\def\PYGZrb{]}
\makeatother

\renewcommand\PYGZsq{\textquotesingle}

\begin{document}

\maketitle
\tableofcontents
\phantomsection\label{code::doc}

\index{an\_example\_pypi\_project (module)}
A pypi demonstration vehicle.


\chapter{useful \#1 -- auto members}
\label{code:documentation-for-the-code}\label{code:module-an_example_pypi_project}\label{code:useful-1-auto-members}
This is something I want to say that is not in the docstring.
\phantomsection\label{code:module-an_example_pypi_project.useful_1}\index{an\_example\_pypi\_project.useful\_1 (module)}\phantomsection\label{code:module-useful_1}\index{useful\_1 (module)}\index{MyPublicClass (class in an\_example\_pypi\_project.useful\_1)}

\begin{fulllineitems}
\phantomsection\label{code:an_example_pypi_project.useful_1.MyPublicClass}\pysiglinewithargsret{\strong{class }\code{an\_example\_pypi\_project.useful\_1.}\bfcode{MyPublicClass}}{\emph{foo}, \emph{bar='baz'}}{}
We use this as a public class example class.

You never call this class before calling \code{public\_fn\_with\_sphinxy\_docstring()}.

\begin{notice}{note}{Note:}
An example of intersphinx is this: you \textbf{cannot} use \href{https://docs.python.org/library/pickle.html\#module-pickle}{\code{pickle}} on this class.
\end{notice}
\index{get\_foobar() (an\_example\_pypi\_project.useful\_1.MyPublicClass method)}

\begin{fulllineitems}
\phantomsection\label{code:an_example_pypi_project.useful_1.MyPublicClass.get_foobar}\pysiglinewithargsret{\bfcode{get\_foobar}}{\emph{foo}, \emph{bar=True}}{}
This gets the foobar

This really should have a full function definition, but I am too lazy.

\begin{Verbatim}[commandchars=\\\{\}]
\PYG{g+gp}{\PYGZgt{}\PYGZgt{}\PYGZgt{} }\PYG{k}{print} \PYG{n}{get\PYGZus{}foobar}\PYG{p}{(}\PYG{l+m+mi}{10}\PYG{p}{,} \PYG{l+m+mi}{20}\PYG{p}{)}
\PYG{g+go}{30}
\PYG{g+gp}{\PYGZgt{}\PYGZgt{}\PYGZgt{} }\PYG{k}{print} \PYG{n}{get\PYGZus{}foobar}\PYG{p}{(}\PYG{l+s+s1}{\PYGZsq{}}\PYG{l+s+s1}{a}\PYG{l+s+s1}{\PYGZsq{}}\PYG{p}{,} \PYG{l+s+s1}{\PYGZsq{}}\PYG{l+s+s1}{b}\PYG{l+s+s1}{\PYGZsq{}}\PYG{p}{)}
\PYG{g+go}{ab}
\end{Verbatim}

Isn't that what you want?

\end{fulllineitems}


\end{fulllineitems}

\index{public\_fn\_with\_googley\_docstring() (in module an\_example\_pypi\_project.useful\_1)}

\begin{fulllineitems}
\phantomsection\label{code:an_example_pypi_project.useful_1.public_fn_with_googley_docstring}\pysiglinewithargsret{\code{an\_example\_pypi\_project.useful\_1.}\bfcode{public\_fn\_with\_googley\_docstring}}{\emph{name}, \emph{state=None}}{}
This function does something.
\begin{description}
\item[{Args:}] \leavevmode
name (str):  The name to use.

\item[{Kwargs:}] \leavevmode
state (bool): Current state to be in.

\item[{Returns:}] \leavevmode
int.  The return code:

\begin{Verbatim}[commandchars=\\\{\}]
0 \PYGZhy{}\PYGZhy{} Success!
1 \PYGZhy{}\PYGZhy{} No good.
2 \PYGZhy{}\PYGZhy{} Try again.
\end{Verbatim}

\item[{Raises:}] \leavevmode
AttributeError, KeyError

\end{description}

A really great idea.  A way you might use me is

\begin{Verbatim}[commandchars=\\\{\}]
\PYG{g+gp}{\PYGZgt{}\PYGZgt{}\PYGZgt{} }\PYG{k}{print} \PYG{n}{public\PYGZus{}fn\PYGZus{}with\PYGZus{}googley\PYGZus{}docstring}\PYG{p}{(}\PYG{n}{name}\PYG{o}{=}\PYG{l+s+s1}{\PYGZsq{}}\PYG{l+s+s1}{foo}\PYG{l+s+s1}{\PYGZsq{}}\PYG{p}{,} \PYG{n}{state}\PYG{o}{=}\PYG{n+nb+bp}{None}\PYG{p}{)}
\PYG{g+go}{0}
\end{Verbatim}

BTW, this always returns 0.  \textbf{NEVER} use with \code{MyPublicClass}.

\end{fulllineitems}

\index{public\_fn\_with\_sphinxy\_docstring() (in module an\_example\_pypi\_project.useful\_1)}

\begin{fulllineitems}
\phantomsection\label{code:an_example_pypi_project.useful_1.public_fn_with_sphinxy_docstring}\pysiglinewithargsret{\code{an\_example\_pypi\_project.useful\_1.}\bfcode{public\_fn\_with\_sphinxy\_docstring}}{\emph{name}, \emph{state=None}}{}
This function does something.
\begin{quote}\begin{description}
\item[{Parameters}] \leavevmode\begin{itemize}
\item {} 
\textbf{\texttt{name}} (\emph{\texttt{str.}}) -- The name to use.

\item {} 
\textbf{\texttt{state}} (\emph{\texttt{bool.}}) -- Current state to be in.

\end{itemize}

\item[{Returns}] \leavevmode
int -- the return code.

\item[{Raises}] \leavevmode
AttributeError, KeyError

\end{description}\end{quote}

\end{fulllineitems}



\chapter{useful \#2 -- explicit members}
\label{code:useful-2-explicit-members}
This is something I want to say that is not in the docstring.
\phantomsection\label{code:module-an_example_pypi_project.useful_2}\index{an\_example\_pypi\_project.useful\_2 (module)}\phantomsection\label{code:module-useful_1}\index{useful\_1 (module)}\index{public\_fn\_with\_sphinxy\_docstring() (in module an\_example\_pypi\_project.useful\_2)}

\begin{fulllineitems}
\phantomsection\label{code:an_example_pypi_project.useful_2.public_fn_with_sphinxy_docstring}\pysiglinewithargsret{\code{an\_example\_pypi\_project.useful\_2.}\bfcode{public\_fn\_with\_sphinxy\_docstring}}{\emph{name}, \emph{state=None}}{}
This function does something.
\begin{quote}\begin{description}
\item[{Parameters}] \leavevmode\begin{itemize}
\item {} 
\textbf{\texttt{name}} (\emph{\texttt{str.}}) -- The name to use.

\item {} 
\textbf{\texttt{state}} (\emph{\texttt{bool.}}) -- Current state to be in.

\end{itemize}

\item[{Returns}] \leavevmode
int -- the return code.

\item[{Raises}] \leavevmode
AttributeError, KeyError

\end{description}\end{quote}

\end{fulllineitems}

\index{\_private\_fn\_with\_docstring() (in module an\_example\_pypi\_project.useful\_2)}

\begin{fulllineitems}
\phantomsection\label{code:an_example_pypi_project.useful_2._private_fn_with_docstring}\pysiglinewithargsret{\code{an\_example\_pypi\_project.useful\_2.}\bfcode{\_private\_fn\_with\_docstring}}{\emph{foo}, \emph{bar='baz'}, \emph{foobarbas=None}}{}
I have a docstring, but won't be imported if you just use \code{:members:}.

\end{fulllineitems}

\index{MyPublicClass (class in an\_example\_pypi\_project.useful\_2)}

\begin{fulllineitems}
\phantomsection\label{code:an_example_pypi_project.useful_2.MyPublicClass}\pysiglinewithargsret{\strong{class }\code{an\_example\_pypi\_project.useful\_2.}\bfcode{MyPublicClass}}{\emph{foo}, \emph{bar='baz'}}{}
We use this as a public class example class.

You never call this class before calling \code{public\_fn\_with\_sphinxy\_docstring()}.

\begin{notice}{note}{Note:}
An example of intersphinx is this: you \textbf{cannot} use \href{https://docs.python.org/library/pickle.html\#module-pickle}{\code{pickle}} on this class.
\end{notice}
\index{\_get\_baz() (an\_example\_pypi\_project.useful\_2.MyPublicClass method)}

\begin{fulllineitems}
\phantomsection\label{code:an_example_pypi_project.useful_2.MyPublicClass._get_baz}\pysiglinewithargsret{\bfcode{\_get\_baz}}{\emph{baz=None}}{}
A private function to get baz.

This really should have a full function definition, but I am too lazy.

\end{fulllineitems}

\index{get\_foobar() (an\_example\_pypi\_project.useful\_2.MyPublicClass method)}

\begin{fulllineitems}
\phantomsection\label{code:an_example_pypi_project.useful_2.MyPublicClass.get_foobar}\pysiglinewithargsret{\bfcode{get\_foobar}}{\emph{foo}, \emph{bar=True}}{}
This gets the foobar

This really should have a full function definition, but I am too lazy.

\begin{Verbatim}[commandchars=\\\{\}]
\PYG{g+gp}{\PYGZgt{}\PYGZgt{}\PYGZgt{} }\PYG{k}{print} \PYG{n}{get\PYGZus{}foobar}\PYG{p}{(}\PYG{l+m+mi}{10}\PYG{p}{,} \PYG{l+m+mi}{20}\PYG{p}{)}
\PYG{g+go}{30}
\PYG{g+gp}{\PYGZgt{}\PYGZgt{}\PYGZgt{} }\PYG{k}{print} \PYG{n}{get\PYGZus{}foobar}\PYG{p}{(}\PYG{l+s+s1}{\PYGZsq{}}\PYG{l+s+s1}{a}\PYG{l+s+s1}{\PYGZsq{}}\PYG{p}{,} \PYG{l+s+s1}{\PYGZsq{}}\PYG{l+s+s1}{b}\PYG{l+s+s1}{\PYGZsq{}}\PYG{p}{)}
\PYG{g+go}{ab}
\end{Verbatim}

Isn't that what you want?

\end{fulllineitems}


\end{fulllineitems}



\chapter{Indices and tables}
\label{code:indices-and-tables}\begin{itemize}
\item {} 
\DUspan{xref,std,std-ref}{genindex}

\item {} 
\DUspan{xref,std,std-ref}{modindex}

\item {} 
\DUspan{xref,std,std-ref}{search}

\end{itemize}


\renewcommand{\indexname}{Python Module Index}
\begin{theindex}
\def\bigletter#1{{\Large\sffamily#1}\nopagebreak\vspace{1mm}}
\bigletter{a}
\item {\texttt{an\_example\_pypi\_project}}, \pageref{code:module-an_example_pypi_project}
\item {\texttt{an\_example\_pypi\_project.useful\_1}}, \pageref{code:module-an_example_pypi_project.useful_1}
\item {\texttt{an\_example\_pypi\_project.useful\_2}}, \pageref{code:module-an_example_pypi_project.useful_2}
\indexspace
\bigletter{u}
\item {\texttt{useful\_1}} \emph{(Unix, Windows)}, \pageref{code:module-useful_1}
\end{theindex}

\renewcommand{\indexname}{Index}
\printindex
\end{document}
